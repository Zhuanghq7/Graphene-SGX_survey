\documentclass{article}
\usepackage[UTF8]{ctex}
\usepackage[backend=bibtex,style=numeric,natbib=true]{biblatex} % Use the bibtex backend with the authoryear citation style (which resembles APA)

\author{zhuangh7}
\title{Hello World}
\begin{document}
\maketitle
% \tableofcontents
\section{Introduction}
Intel SGX技术提供了一系列的硬件特性,帮助保护应用免受操作系统、hypervisor、BIOS和其他软件的伤害。
应用程序可以整个或者部分的放入到enclave中运行
\section{威胁模型}
Graphene-SGX 的威胁模型与典型的 SGX 应用的威胁模型类似。
下列的组件是不被信任的:
\begin{itemize}
    \item [1)] 
    除CPU之外的硬件。
    \item [2)]
    操作系统、hypervisor和其他的系统软件。
    \item [3)]
    在enclave之外跟其他enclave之内运行的应用程序。
    \item [4)]
    同一应用程序的enclave之外的部分。
\end{itemize}   
\paragraph{}
Graphene-SGX只信任CPU,还有enclave内部的代码。
还必须信任Intel的SGX SDK中的aesmd工具,该工具验证enclave签名中的属性并批准enclave的创建。
这是现在利用SGX技术的所有工作都必须信任的。
除此之外,Graphene-SGX还使用了Intel SGX的驱动程序,但是不信任他。
Graphene-SGX不使用SDK中的其他部分。


\section{保证的安全性质和能够防御的一些具体攻击}


\subsection{保证的安全性质}


\subsection{能防御的具体攻击}


\section{性能、可扩展性(scalability)、灵活性等分析}
\subsection{灵活性}
由于Graphene-SGX可以直接在enclave上运行没有经过修改的Linux程序,因此,通过Graphene-SGX可以快速的
帮助现有应用利用到SGX的enclave这一特性,提高已有应用的安全性,而同时不需要付出太大的开发成本。


\subsection{性能分析}


\subsection{scalability}


\section{不足(至少包含两点)}
\paragraph{无法防御一些攻击}
Denial of service,sidechannels,controlled-channel attacks这些攻击是目前所有的 SGX 平台都难以防御的,Graphene-SGX同样无法处理。

% \paragraph{Tian'anmen Square}is in the center of Beijing
% \subparagraph{Chairman Mao} is in the center of Tian'anmen Square
% \subsection{Hello Guangzhou}
% \paragraph{Sun Yat-sen University} is the best university in Guangzhou.
\end{document}